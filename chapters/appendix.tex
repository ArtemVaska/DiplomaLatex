\clearpage
\section{Приложение}


\begin{longtable}[c]{|c|c|c|c|c|}
\caption{Сводная таблица с характеристикой кассетного интрона для таксономической группы Ecdysozoa.
Сортировка по возрастанию количества нуклеотидов до стоп-кодона в кассетной интроне.}
\label{tab:Ecdysozoa}\\
\hline
\textbf{\begin{tabular}[c]{@{}c@{}}Название\\ организма\end{tabular}} &
  \textbf{\begin{tabular}[c]{@{}c@{}}Кол-во\\ нуклеотидов\\ до стоп-кодона\\ в интроне\end{tabular}} &
  \textbf{\begin{tabular}[c]{@{}c@{}}Длина\\ 1-го экзона\\ в кассете\end{tabular}} &
  \textbf{\begin{tabular}[c]{@{}c@{}}Длина\\ кассетного\\ интрона\end{tabular}} &
  \textbf{\begin{tabular}[c]{@{}c@{}}Длина\\ 2-го экзона\\ в кассете\end{tabular}} \\ \hline
\endfirsthead
%
\endhead
%
\hline
\endfoot
%
\endlastfoot
%
\textit{Trichinella spiralis}             & 1    & 83  & 417  & 37 \\
\textit{Priapulus caudatus}               & 1    & 110 & 2114 & 37 \\
\textit{Galendromus occidentalis}         & 1    & 110 & 1491 & 37 \\
\textit{Ixodes scapularis}                & 1    & 110 & 3567 & 37 \\
\textit{Limulus polyphemus}               & 1    & 110 & 915  & 37 \\
\textit{Parasteatoda tepidariorum}        & 1    & 110 & 1725 & 37 \\
\textit{Cryptotermes secundus}            & 1    & 110 & 4335 & 37 \\
\textit{Maniola hyperantus}               & 1    & 110 & 920  & 37 \\
\textit{Cimex lectularius}                & 1    & 110 & 4437 & 37 \\
\textit{Vespa mandarinia}                 & 1    & 113 & 379  & 37 \\
\textit{Zerene cesonia}                   & 1    & 110 & 1162 & 37 \\
\textit{Pararge aegeria}                  & 1    & 110 & 2657 & 37 \\
\textit{Myzus persicae}                   & 1    & 107 & 772  & 37 \\
\textit{Halyomorpha halys}                & 1    & 110 & 7270 & 37 \\
\textit{Diuraphis noxia}                  & 1    & 107 & 742  & 37 \\
\textit{Sipha flava}                      & 1    & 107 & 58   & 37 \\
\textit{Manduca sexta}                    & 1    & 110 & 1796 & 37 \\
\textit{Apis laboriosa}                   & 1    & 113 & 1254 & 37 \\
\textit{Orussus abietinus}                & 1    & 113 & 74   & 37 \\
\textit{Danaus plexippus}                 & 1    & 110 & 1009 & 37 \\
\textit{Colletes gigas}                   & 1    & 113 & 379  & 37 \\
\textit{Ostrinia furnacalis}              & 1    & 110 & 1946 & 37 \\
\textit{Vespa crabro}                     & 1    & 113 & 381  & 37 \\
\textit{Venturia canescens}               & 1    & 113 & 621  & 37 \\
\textit{Papilio polytes}                  & 1    & 110 & 1674 & 37 \\
\textit{Vespa velutina}                   & 1    & 113 & 377  & 37 \\
\textit{Cephus cinctus}                   & 1    & 113 & 75   & 37 \\
\textit{Bombus pyrosoma}                  & 1    & 113 & 244  & 37 \\
\textit{Papilio xuthus}                   & 1    & 110 & 999  & 37 \\
\textit{Vanessa tameamea}                 & 1    & 110 & 2352 & 37 \\
\textit{Megalopta genalis}                & 1    & 113 & 373  & 37 \\
\textit{Vespula pensylvanica}             & 1    & 113 & 363  & 37 \\
\textit{Leptopilina heterotoma}           & 1    & 113 & 921  & 37 \\
\textit{Acromyrmex echinatior}            & 1    & 113 & 438  & 37 \\
\textit{Aphidius gifuensis}               & 1    & 113 & 240  & 37 \\
\textit{Polistes fuscatus}                & 1    & 113 & 400  & 37 \\
\textit{Dirofilaria immitis}              & 7    & 98  & 248  & 37 \\
\textit{Odontomachus brunneus}            & 10   & 113 & 498  & 37 \\
\textit{Diploscapter pachys}              & 10   & 110 & 662  & 37 \\
\textit{Bactrocera dorsalis}              & 13   & 110 & 1808 & 37 \\
\textit{Drosophila melanogaster}          & 13   & 110 & 1602 & 37 \\
\textit{Ceratitis capitata}               & 19   & 110 & 2023 & 37 \\
\textit{Pediculus humanus corporis}       & 19   & 110 & 631  & 37 \\
\textit{Aphelenchoides avenae}            & 19   & 110 & 441  & 37 \\
\textit{Litomosoides sigmodontis}         & 19   & 110 & 242  & 37 \\
\textit{Acanthocheilonema viteae}         & 19   & 110 & 225  & 37 \\
\textit{Aethina tumida}                   & 19   & 110 & 1729 & 37 \\
\textit{Lepeophtheirus salmonis}          & 22   & 110 & 1555 & 37 \\
\textit{Anoplophora glabripennis}         & 22   & 110 & 3664 & 37 \\
\textit{Varroa jacobsoni}                 & 22   & 110 & 3077 & 37 \\
\textit{Varroa destructor}                & 22   & 110 & 3077 & 37 \\
\textit{Thelazia callipaeda}              & 25   & 110 & 209  & 37 \\
\textit{Bursaphelenchus xylophilus}       & 25   & 110 & 638  & 37 \\
\textit{Acyrthosiphon pisum}              & 28   & 107 & 68   & 37 \\
\textit{Anisakis simplex}                 & 30   & 219 & 665  & 37 \\
\textit{Tetranychus urticae}              & 31   & 122 & 648  & 37 \\
\textit{Homarus americanus}               & 31   & 110 & 9821 & 37 \\
\textit{Bursaphelenchus okinawaensis}     & 37   & 110 & 593  & 37 \\
\textit{Globodera pallida}                & 43   & 113 & 47   & 37 \\
\textit{Amphibalanus amphitrite}          & 73   & 110 & 369  & 37 \\
\textit{Cotesia glomerata}                & 73   & 116 & 236  & 37 \\
\textit{Caenorhabditis angaria}           & 79   & 110 & 96   & 37 \\
\textit{Onchocerca ochengi}               & 88   & 110 & 243  & 37 \\
\textit{Brugia pahangi}                   & 91   & 110 & 232  & 37 \\
\textit{Ditylenchus destructor}           & 97   & 307 & 1167 & 37 \\
\textit{Mesorhabditis belari}             & 97   & 110 & 147  & 37 \\
\textit{Melanaphis sacchari}              & 97   & 107 & 71   & 37 \\
\textit{Enterobius vermicularis}          & 100  & 110 & 195  & 37 \\
\textit{Pristionchus mayeri}              & 103  & 110 & 131  & 37 \\
\textit{Cercopithifilaria johnstoni}      & 103  & 110 & 238  & 37 \\
\textit{Steinernema carpocapsae}          & 106  & 110 & 131  & 37 \\
\textit{Wuchereria bancrofti}             & 106  & 125 & 242  & 37 \\
\textit{Parelaphostrongylus tenuis}       & 112  & 110 & 228  & 37 \\
\textit{Toxocara canis}                   & 115  & 110 & 1062 & 37 \\
\textit{Necator americanus}               & 136  & 110 & 243  & 37 \\
\textit{Brugia malayi}                    & 139  & 110 & 243  & 37 \\
\textit{Caenorhabditis auriculariae}      & 145  & 110 & 156  & 37 \\
\textit{Auanema sp. JU1783}               & 145  & 110 & 80   & 37 \\
\textit{Pristionchus entomophagus}        & 151  & 110 & 154  & 37 \\
\textit{Steinernema hermaphroditum}       & 157  & 110 & 131  & 37 \\
\textit{Caenorhabditis brenneri}          & 175  & 110 & 130  & 37 \\
\textit{Angiostrongylus cantonensis}      & 181  & 110 & 213  & 37 \\
\textit{Dictyocaulus viviparus}           & 190  & 110 & 832  & 37 \\
\textit{Caenorhabditis elegans}           & 193  & 110 & 106  & 37 \\
\textit{Cooperia oncophora}               & 205  & 110 & 215  & 37 \\
\textit{Caenorhabditis sp. 36 PRJEB53466} & 205  & 110 & 133  & 37 \\
\textit{Caenorhabditis nigoni}            & 214  & 110 & 142  & 37 \\
\textit{Pristionchus pacificus}           & 214  & 110 & 251  & 37 \\
\textit{Trichostrongylus colubriformis}   & 214  & 110 & 224  & 37 \\
\textit{Caenorhabditis briggsae}          & 217  & 110 & 145  & 37 \\
\textit{Cylicocyclus nassatus}            & 229  & 110 & 239  & 37 \\
\textit{Haemonchus contortus}             & 304  & 110 & 220  & 37 \\
\textit{Caenorhabditis bovis}             & 316  & 110 & 235  & 37 \\
\textit{Nippostrongylus brasiliensis}     & 316  & 110 & 235  & 37 \\
\textit{Dracunculus medinensis}           & 334  & 110 & 122  & 37 \\
\textit{Mesorhabditis spiculigera}        & 376  & 110 & 173  & 37 \\
\textit{Pollicipes pollicipes}            & 436  & 110 & 367  & 37 \\
\textit{Rhopalosiphum maidis}             & 1345 & 107 & 69   & 37 \\ \hline
\end{longtable}


\begin{longtable}[c]{|c|c|c|c|c|}
\caption{Сводная таблица с характеристикой кассетного интрона для таксономической группы Spiralia.
Сортировка по возрастанию количества нуклеотидов до стоп-кодона в ``кассетном`` интроне.}
\label{tab:Spiralia}\\
\hline
\textbf{\begin{tabular}[c]{@{}c@{}}Название\\ организма\end{tabular}} &
  \textbf{\begin{tabular}[c]{@{}c@{}}Кол-во\\ нуклеотидов\\ до стоп-кодона\\ в интроне\end{tabular}} &
  \textbf{\begin{tabular}[c]{@{}c@{}}Длина\\ 1-го экзона\\ в кассете\end{tabular}} &
  \textbf{\begin{tabular}[c]{@{}c@{}}Длина\\ кассетного\\ интрона\end{tabular}} &
  \textbf{\begin{tabular}[c]{@{}c@{}}Длина\\ 2-го экзона\\ в кассете\end{tabular}} \\ \hline
\endfirsthead
%
\endhead
%
\hline
\endfoot
%
\endlastfoot
%
\textit{Schistosoma haematobium}   & 1  & 239 & 652   & 37 \\
\textit{Magallana gigas}           & 1  & 110 & 1537  & 37 \\
\textit{Mya arenaria}              & 1  & 110 & 1727  & 37 \\
\textit{Crassostrea virginica}     & 1  & 110 & 1613  & 37 \\
\textit{Aplysia californica}       & 1  & 221 & 4146  & 37 \\
\textit{Gigantopelta aegis}        & 1  & 110 & 1869  & 37 \\
\textit{Mercenaria mercenaria}     & 1  & 110 & 1690  & 37 \\
\textit{Dreissena polymorpha}      & 1  & 110 & 2207  & 37 \\
\textit{Ruditapes philippinarum}   & 1  & 110 & 1646  & 37 \\
\textit{Mactra antiquata}          & 1  & 110 & 2319  & 37 \\
\textit{Mytilus coruscus}          & 1  & 110 & 1234  & 37 \\
\textit{Potamilus streckersoni}    & 1  & 110 & 4567  & 37 \\
\textit{Saccostrea echinata}       & 1  & 110 & 1556  & 37 \\
\textit{Mytilus edulis}            & 1  & 110 & 1360  & 37 \\
\textit{Mytilus trossulus}         & 1  & 110 & 1357  & 37 \\
\textit{Pecten maximus}            & 1  & 110 & 5000  & 37 \\
\textit{Ostrea edulis}             & 1  & 110 & 1643  & 37 \\
\textit{Mizuhopecten yessoensis}   & 1  & 110 & 4836  & 37 \\
\textit{Saccostrea cuccullata}     & 1  & 110 & 1706  & 37 \\
\textit{Ylistrum balloti}          & 1  & 110 & 4649  & 37 \\
\textit{Argopecten irradians}      & 1  & 110 & 5057  & 37 \\
\textit{Magallana angulata}        & 1  & 110 & 1534  & 37 \\
\textit{Mytilus californianus}     & 1  & 110 & 1248  & 37 \\
\textit{Pinctada imbricata}        & 1  & 110 & 4144  & 37 \\
\textit{Haliotis asinina}          & 1  & 110 & 2375  & 37 \\
\textit{Sinanodonta woodiana}      & 1  & 110 & 4580  & 37 \\
\textit{Haliotis cracherodii}      & 1  & 110 & 2506  & 37 \\
\textit{Haliotis rufescens}        & 1  & 110 & 2505  & 37 \\
\textit{Patella caerulea}          & 1  & 110 & 1362  & 37 \\
\textit{Patella vulgata}           & 1  & 110 & 1384  & 37 \\
\textit{Lymnaea stagnalis}         & 1  & 221 & 2705  & 37 \\
\textit{Batillaria attramentaria}  & 1  & 110 & 8614  & 37 \\
\textit{Schistosoma turkestanicum} & 1  & 239 & 905   & 37 \\
\textit{Paragonimus westermani}    & 1  & 239 & 13971 & 37 \\
\textit{Pomacea canaliculata}      & 1  & 56  & 255   & 37 \\
\textit{Bradybaena similaris}      & 1  & 221 & 3811  & 37 \\
\textit{Elysia crispata}           & 1  & 221 & 8063  & 37 \\
\textit{Elysia chlorotica}         & 1  & 221 & 7182  & 37 \\
\textit{Bulinus truncatus}         & 1  & 221 & 1873  & 37 \\
\textit{Biomphalaria pfeifferi}    & 1  & 221 & 1885  & 37 \\
\textit{Biomphalaria glabrata}     & 1  & 221 & 1889  & 37 \\
\textit{Schistosoma guineensis}    & 1  & 239 & 652   & 37 \\
\textit{Schistosoma curassoni}     & 1  & 239 & 652   & 37 \\
\textit{Schistosoma bovis}         & 1  & 239 & 652   & 37 \\
\textit{Schistosoma margrebowiei}  & 1  & 239 & 650   & 37 \\
\textit{Schistosoma intercalatum}  & 1  & 239 & 652   & 37 \\
\textit{Schistosoma rodhaini}      & 1  & 239 & 671   & 37 \\
\textit{Schistosoma japonicum}     & 1  & 239 & 847   & 37 \\
\textit{Clonorchis sinensis}       & 1  & 242 & 6006  & 37 \\
\textit{Hydatigera taeniaeformis}  & 1  & 242 & 375   & 37 \\
\textit{Taenia crassiceps}         & 1  & 242 & 278   & 37 \\
\textit{Taenia asiatica}           & 1  & 242 & 480   & 37 \\
\textit{Heterobilharzia americana} & 1  & 239 & 2163  & 37 \\
\textit{Trichobilharzia szidati}   & 1  & 239 & 1336  & 37 \\
\textit{Trichobilharzia regenti}   & 1  & 239 & 996   & 37 \\
\textit{Opisthorchis felineus}     & 1  & 242 & 14603 & 37 \\
\textit{Rodentolepis nana}         & 1  & 242 & 222   & 37 \\
\textit{Calicophoron daubneyi}     & 1  & 239 & 4214  & 37 \\
\textit{Taenia solium}             & 1  & 242 & 480   & 37 \\
\textit{Echinococcus granulosus}   & 1  & 242 & 521   & 37 \\
\textit{Fasciola hepatica}         & 1  & 239 & 2631  & 37 \\
\textit{Fasciola gigantica}        & 1  & 239 & 2581  & 37 \\
\textit{Schistosoma mattheei}      & 1  & 239 & 649   & 37 \\
\textit{Fasciolopsis buskii}       & 1  & 239 & 1303  & 37 \\
\textit{Dicrocoelium dendriticum}  & 1  & 239 & 2612  & 37 \\
\textit{Paragonimus heterotremus}  & 1  & 239 & 18219 & 37 \\
\textit{Hymenolepis diminuta}      & 1  & 242 & 224   & 37 \\
\textit{Solemya velum}             & 4  & 110 & 2071  & 37 \\
\textit{Littorina saxatilis}       & 19 & 218 & 6746  & 37 \\ \hline
\end{longtable}


\begin{longtable}[c]{|c|c|c|c|c|}
\caption{Сводная таблица с характеристикой кассетного интрона для таксономической группы Cnidaria.
Сортировка по возрастанию количества нуклеотидов до стоп-кодона в ``кассетном`` интроне.}
\label{tab:Cnidaria}\\
\hline
\textbf{\begin{tabular}[c]{@{}c@{}}Название\\ организма\end{tabular}} &
  \textbf{\begin{tabular}[c]{@{}c@{}}Кол-во\\ нуклеотидов\\ до стоп-кодона\\ в интроне\end{tabular}} &
  \textbf{\begin{tabular}[c]{@{}c@{}}Длина\\ 1-го экзона\\ в кассете\end{tabular}} &
  \textbf{\begin{tabular}[c]{@{}c@{}}Длина\\ кассетного\\ интрона\end{tabular}} &
  \textbf{\begin{tabular}[c]{@{}c@{}}Длина\\ 2-го экзона\\ в кассете\end{tabular}} \\ \hline
\endfirsthead
%
\endhead
%
\hline
\endfoot
%
\endlastfoot
%
\textit{Actinia tenebrosa}       & 10  & 116 & 173 & 37 \\
\textit{Dendronephthya gigantea} & 10  & 116 & 328 & 37 \\
\textit{Nematostella vectensis}  & 25  & 116 & 991 & 37 \\
\textit{Montipora foliosa}       & 31  & 116 & 907 & 37 \\
\textit{Pocillopora verrucosa}   & 34  & 116 & 390 & 37 \\
\textit{Acropora digitifera}     & 40  & 116 & 670 & 37 \\
\textit{Acropora millepora}      & 40  & 116 & 682 & 37 \\
\textit{Acropora muricata}       & 40  & 116 & 679 & 37 \\
\textit{Pocillopora damicornis}  & 46  & 116 & 392 & 37 \\
\textit{Pocillopora meandrina}   & 46  & 116 & 392 & 37 \\
\textit{Porites lutea}           & 61  & 116 & 711 & 37 \\
\textit{Porites evermanni}       & 61  & 116 & 711 & 37 \\
\textit{Exaiptasia diaphana}     & 76  & 86  & 227 & 37 \\
\textit{Xenia sp. Carnegie-2017} & 103 & 116 & 116 & 37 \\ \hline
\end{longtable}


\begin{longtable}[c]{|c|c|c|c|c|}
\caption{Сводная таблица с характеристикой кассетного интрона для таксономической группы Sauropsida.
Сортировка по возрастанию количества нуклеотидов до стоп-кодона в ``кассетном`` интроне.}
\label{tab:Sauropsida}\\
\hline
\textbf{\begin{tabular}[c]{@{}c@{}}Название\\ организма\end{tabular}} &
  \textbf{\begin{tabular}[c]{@{}c@{}}Кол-во\\ нуклеотидов\\ до стоп-кодона\\ в интроне\end{tabular}} &
  \textbf{\begin{tabular}[c]{@{}c@{}}Длина\\ 1-го экзона\\ в кассете\end{tabular}} &
  \textbf{\begin{tabular}[c]{@{}c@{}}Длина\\ кассетного\\ интрона\end{tabular}} &
  \textbf{\begin{tabular}[c]{@{}c@{}}Длина\\ 2-го экзона\\ в кассете\end{tabular}} \\ \hline
\endfirsthead
%
\endhead
%
\hline
\endfoot
%
\endlastfoot
%
\textit{Molothrus aeneus}             & 1    & 110 & 745  & 37 \\
\textit{Taeniopygia guttata}          & 1    & 110 & 443  & 37 \\
\textit{Lonchura striata}             & 1    & 110 & 629  & 37 \\
\textit{Gallus gallus}                & 7    & 110 & 1616 & 37 \\
\textit{Cygnus atratus}               & 25   & 110 & 1257 & 37 \\
\textit{Haliaeetus leucocephalus}     & 25   & 110 & 1375 & 37 \\
\textit{Phalacrocorax carbo}          & 25   & 110 & 1345 & 37 \\
\textit{Grus americana}               & 25   & 110 & 1659 & 37 \\
\textit{Haliaeetus albicilla}         & 25   & 110 & 1378 & 37 \\
\textit{Oxyura jamaicensis}           & 25   & 110 & 1246 & 37 \\
\textit{Anser cygnoides}              & 25   & 110 & 1279 & 37 \\
\textit{Ciconia boyciana}             & 25   & 107 & 1459 & 37 \\
\textit{Anas acuta}                   & 25   & 110 & 1346 & 37 \\
\textit{Astur gentilis}               & 25   & 110 & 1393 & 37 \\
\textit{Aquila chrysaetos chrysaetos} & 25   & 110 & 1375 & 37 \\
\textit{Aythya fuligula}              & 25   & 110 & 1227 & 37 \\
\textit{Struthio camelus}             & 64   & 110 & 1405 & 37 \\
\textit{Chelonia mydas}               & 79   & 110 & 1674 & 37 \\
\textit{Dermochelys coriacea}         & 79   & 110 & 1661 & 37 \\
\textit{Caretta caretta}              & 79   & 110 & 1656 & 37 \\
\textit{Ammospiza caudacuta}          & 82   & 110 & 3942 & 37 \\
\textit{Aphelocoma coerulescens}      & 85   & 110 & 3626 & 37 \\
\textit{Gopherus flavomarginatus}     & 142  & 110 & 1655 & 37 \\
\textit{Chelonoidis abingdonii}       & 142  & 110 & 1645 & 37 \\
\textit{Malaclemys terrapin pileata}  & 142  & 110 & 1652 & 37 \\
\textit{Mauremys mutica}              & 142  & 110 & 1662 & 37 \\
\textit{Mauremys reevesii}            & 142  & 110 & 1661 & 37 \\
\textit{Trachemys scripta elegans}    & 142  & 110 & 1661 & 37 \\
\textit{Chrysemys picta bellii}       & 142  & 110 & 1662 & 37 \\
\textit{Emys orbicularis}             & 142  & 110 & 1650 & 37 \\
\textit{Alligator sinensis}           & 148  & 110 & 1497 & 37 \\
\textit{Alligator mississippiensis}   & 148  & 110 & 1618 & 37 \\
\textit{Caloenas nicobarica}          & 184  & 110 & 1245 & 37 \\
\textit{Rissa tridactyla}             & 205  & 110 & 1388 & 37 \\
\textit{Terrapene triunguis}          & 211  & 110 & 1662 & 37 \\
\textit{Emydura macquarii macquarii}  & 223  & 110 & 1647 & 37 \\
\textit{Catharus ustulatus}           & 241  & 110 & 3252 & 37 \\
\textit{Gopherus evgoodei}            & 301  & 110 & 1639 & 37 \\
\textit{Strigops habroptila}          & 457  & 110 & 1317 & 37 \\
\textit{Neopsephotus bourkii}         & 502  & 110 & 1245 & 37 \\
\textit{Melopsittacus undulatus}      & 517  & 110 & 1257 & 37 \\
\textit{Apteryx rowi}                 & 541  & 110 & 1359 & 37 \\
\textit{Apteryx mantelli}             & 541  & 110 & 1359 & 37 \\
\textit{Dromaius novaehollandiae}     & 553  & 110 & 1365 & 37 \\
\textit{Chroicocephalus ridibundus}   & 562  & 110 & 1373 & 37 \\
\textit{Pezoporus wallicus}           & 568  & 110 & 1328 & 37 \\
\textit{Pezoporus flaviventris}       & 568  & 110 & 1328 & 37 \\
\textit{Rhea pennata}                 & 568  & 110 & 1348 & 37 \\
\textit{Pezoporus occidentalis}       & 568  & 110 & 1319 & 37 \\
\textit{Pelodiscus sinensis}          & 640  & 110 & 1643 & 37 \\
\textit{Phaenicophaeus curvirostris}  & 892  & 110 & 2155 & 37 \\
\textit{Camarhynchus parvulus}        & 1360 & 110 & 2456 & 37 \\
\textit{Vidua chalybeata}             & 1519 & 110 & 678  & 37 \\ \hline
\end{longtable}


\begin{longtable}[c]{|c|c|c|c|c|}
\caption{Сводная таблица с характеристикой кассетного интрона для таксономической группы Amphibia.
Сортировка по возрастанию количества нуклеотидов до стоп-кодона в ``кассетном`` интроне.}
\label{tab:Amphibia}\\
\hline
\textbf{\begin{tabular}[c]{@{}c@{}}Название\\ организма\end{tabular}} &
  \textbf{\begin{tabular}[c]{@{}c@{}}Кол-во\\ нуклеотидов\\ до стоп-кодона\\ в интроне\end{tabular}} &
  \textbf{\begin{tabular}[c]{@{}c@{}}Длина\\ 1-го экзона\\ в кассете\end{tabular}} &
  \textbf{\begin{tabular}[c]{@{}c@{}}Длина\\ кассетного\\ интрона\end{tabular}} &
  \textbf{\begin{tabular}[c]{@{}c@{}}Длина\\ 2-го экзона\\ в кассете\end{tabular}} \\ \hline
\endfirsthead
%
\endhead
%
\hline
\endfoot
%
\endlastfoot
%
Ambystoma mexicanum     & 1   & 110 & 10340 & 37 \\
Pelobates fuscus        & 1   & 110 & 2424  & 37 \\
Bufo bufo               & 7   & 110 & 3002  & 37 \\
Bufo gargarizans        & 7   & 110 & 2879  & 37 \\
Hyperolius riggenbachi  & 10  & 110 & 3902  & 37 \\
Rana temporaria         & 10  & 110 & 3036  & 37 \\
Pseudophryne corroboree & 19  & 110 & 3561  & 37 \\
Spea bombifrons         & 25  & 110 & 2840  & 37 \\
Engystomops pustulosus  & 25  & 110 & 2004  & 37 \\
Nanorana parkeri        & 25  & 110 & 3038  & 37 \\
Hyla sarda              & 25  & 110 & 3029  & 37 \\
Pyxicephalus adspersus  & 25  & 110 & 2917  & 37 \\
Ranitomeya imitator     & 37  & 110 & 2650  & 37 \\
Xenopus tropicalis      & 46  & 110 & 2596  & 37 \\
Xenopus laevis          & 52  & 110 & 3791  & 37 \\
Geotrypetes seraphini   & 55  & 110 & 3065  & 37 \\
Rhinatrema bivittatum   & 103 & 110 & 4053  & 37 \\
Pleurodeles waltl       & 151 & 110 & 3245  & 37 \\
Microcaecilia unicolor  & 187 & 110 & 2784  & 37 \\ \hline
\end{longtable}


\begin{longtable}[c]{|c|c|c|c|c|}
\caption{Сводная таблица с характеристикой кассетного интрона для таксономической группы Lepidosauria.
Сортировка по возрастанию количества нуклеотидов до стоп-кодона в ``кассетном`` интроне.}
\label{tab:Lepidosauria}\\
\hline
\textbf{\begin{tabular}[c]{@{}c@{}}Название\\ организма\end{tabular}} &
  \textbf{\begin{tabular}[c]{@{}c@{}}Кол-во\\ нуклеотидов\\ до стоп-кодона\\ в интроне\end{tabular}} &
  \textbf{\begin{tabular}[c]{@{}c@{}}Длина\\ 1-го экзона\\ в кассете\end{tabular}} &
  \textbf{\begin{tabular}[c]{@{}c@{}}Длина\\ кассетного\\ интрона\end{tabular}} &
  \textbf{\begin{tabular}[c]{@{}c@{}}Длина\\ 2-го экзона\\ в кассете\end{tabular}} \\ \hline
\endfirsthead
%
\endhead
%
\hline
\endfoot
%
\endlastfoot
%
\textit{Python bivittatus}             & 1 & 110 & 2374 & 37 \\
\textit{Notechis scutatus}             & 1 & 110 & 2507 & 37 \\
\textit{Pseudonaja textilis}           & 1 & 110 & 2519 & 37 \\
\textit{Anolis sagrei}                 & 1 & 110 & 4667 & 37 \\
\textit{Pituophis catenifer annectens} & 1 & 110 & 2420 & 37 \\
\textit{Lacerta agilis}                & 1 & 110 & 2499 & 37 \\
\textit{Candoia aspera}                & 1 & 110 & 2293 & 37 \\
\textit{Sphaerodactylus townsendi}     & 1 & 110 & 2825 & 37 \\
\textit{Thamnophis elegans}            & 1 & 110 & 2426 & 37 \\
\textit{Ahaetulla prasina}             & 1 & 110 & 2432 & 37 \\
\textit{Gekko japonicus}               & 1 & 110 & 2924 & 37 \\
\textit{Crotalus tigris}               & 1 & 110 & 3091 & 37 \\
\textit{Pogona vitticeps}              & 1 & 110 & 2746 & 37 \\
\textit{Podarcis raffonei}             & 1 & 110 & 2495 & 37 \\
\textit{Protobothrops mucrosquamatus}  & 1 & 110 & 3264 & 37 \\
\textit{Varanus komodoensis}           & 1 & 110 & 2658 & 37 \\
\textit{Pantherophis guttatus}         & 1 & 110 & 2411 & 37 \\
\textit{Elgaria multicarinata webbii}  & 1 & 110 & 2800 & 37 \\
\textit{Rhineura floridana}            & 1 & 110 & 2581 & 37 \\
\textit{Podarcis muralis}              & 1 & 110 & 2506 & 37 \\
\textit{Heteronotia binoei}            & 1 & 110 & 3002 & 37 \\
\textit{Anolis carolinensis}           & 1 & 110 & 4026 & 37 \\
\textit{Erythrolamprus reginae}        & 1 & 110 & 2638 & 37 \\
\textit{Sceloporus undulatus}          & 1 & 110 & 2380 & 37 \\
\textit{Eublepharis macularius}        & 1 & 110 & 2577 & 37 \\
\textit{Euleptes europaea}             & 1 & 110 & 2901 & 37 \\
\textit{Hemicordylus capensis}         & 1 & 110 & 2830 & 37 \\
\textit{Zootoca vivipara}              & 1 & 110 & 2516 & 37 \\ \hline
\end{longtable}
