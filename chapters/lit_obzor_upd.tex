\clearpage
\section{Обзор литературы}

\subsection{Механизмы усложнения организации генома}

Увеличение разнообразия транскриптома и протеома у эукариот во многом достигается не только за счет классического эксцизионного сплайсинга, но и за счет ряда альтернативных механизмов обработки пре-мРНК.
В частности, альтернативный сплайсинг позволяет из одного транскрипта формировать несколько зрелых мРНК, отличающихся включением или исключением отдельных экзонов и участков~\cite{Mamon2019,Juneau2006}.

Одним из ключевых вариантов такого процесса является удержание интронов (intron retention, IR), когда интрон не удаляется и остается в составе зрелой мРНК.
Часто подобное сохранение интрона приводит к появлению в получившемся транскрипте преждевременных стоп-кодонов (PTC), что запускает нонсенс-опосредованный распад (NMD).
Тем не менее в ряде случаев внутри интрона формируется устойчивая вторичная структура, препятствующая связыванию факторов NMD и позволяющая таким транскриптам не подвергаться деградации, а при необходимости кодировать укороченные, но функционально активные белки~\cite{Mamon2013,Jo2015,Kalyna2012}.

Кроме IR, альтернативный сплайсинг включает пропуск экзонов, использование альтернативных сайтов на 5′- и 3′-концах и кассетное включение/исключение блоков экзонов.
В совокупности эти механизмы значительно расширяют репертуар возможных изоформ без необходимости увеличения числа генов.
Например, у многих многоклеточных организмов до 95\% генов подвергаются хотя бы одному типу альтернативного сплайсинга, что подчеркивает его эволюционную важность~\cite{Mamon2019}.

Таким образом, именно через комбинирование альтернативных способов сплайсинга, особенно удержания интронов, эукариоты получают мощный инструмент транскрипционной и белковой вариативности, что способствует адаптации и усложнению биологических процессов.


\subsection{Значимость интронов}

Традиционно интроны воспринимались лишь как «ненужные» вставки, но современные исследования убедительно показывают, что их функции выходят далеко за рамки простой ``пустоты``.
Во-первых, наличие интронных последовательностей может значительно усиливать уровень экспрессии генов.
Эксперименты на клеточных системах SV40, дрожжей \textit{Saccharomyces cerevisiae} и млекопитающих демонстрируют, что удаление ключевых интронов приводит к резкому снижению эффективности транскрипции и трансляции~\cite{Gruss1979,Juneau2006}.

Во-вторых, интроны влияют на чувствительность мРНК к нонсенс-опосредованному распаду.
Если интрон попадает в 5′- или 3′-UTR, его присутствие может менять архитектуру сплайсосомного комплекса, корректируя доступность PTC и, соответственно, баланс между сохранением транскрипта и его деградацией через NMD~\cite{Jo2015,Kalyna2012}.

Третья важная роль интронов заключается в транспорте мРНК из ядра в цитоплазму.
Долгое время считалось, что только полностью сплайсированные транскрипты эффективно экспортируются, однако при помощи FISH-методов было показано, что РНК с сохраненными интронами также могут накапливаться в цитоплазме и функционировать там~\cite{Valencia2008,Jo2015}.
Это потребовало пересмотра классических представлений об экспорте мРНК.

Кроме регуляции экспрессии и транспорта, интроны участвуют в организации хроматиновой структуры.
Концевые последовательности интронов образуют участки с пониженной плотностью нуклеосом, что способствует более четкому разделению экзонов и облегчает процесс транскрипции~\cite{Schwartz2009}.

Наконец, интроны могут выполнять таксон- и тканеспецифические функции.
Так, первый интрон гена \textit{oskar} у Drosophila участвует в локализации мРНК в ооците, а длинные интронные вставки могут снижать ``интерференцию Хилла–Робертсона``, улучшая кроссинговер в определенных регионах генома~\cite{Comeron2008}.
Результаты GWAS показывают, что полиморфизмы в интронных областях часто связаны с предрасположенностью к различным заболеваниям человека~\cite{Welter2014}.

Таким образом, интроны выполняют сложные регуляторные функции — от контроля уровня экспрессии до обеспечения оптимальной архитектуры хроматина и тканеспецифической регуляции транскриптов.


\subsection{Семейство генов \textit{Nxf}}

Гены \textit{Nxf} (nuclear export factor) названы по функции их наиболее известного представителя – \textit{Nxf1}, который обеспечивает экспорт большинства мРНК из ядра в цитоплазму.
Распространение этих генов охватывает всех эукариот Opisthokonta, однако их число и структурные особенности заметно различаются между группами.
У грибов обычно присутствует единственная копия \textit{Nxf}, тогда как в геномах растений и некоторых протистов такие гены могут отсутствовать полностью.
У животных же часто встречается от двух до пяти паралогов, что свидетельствует об активных дупликационных процессах в эволюции этого семейства~\cite{Mamon2013}.

Важнейшим элементом всех \textit{Nxf}-генов является ``кассетный`` интрон, помещенный между двумя короткими экзонами (110 и 37 нуклеотидов в каноническом варианте).
При альтернативном сплайсинге этот интрон может сохраняться в зрелой мРНК, неся внутри себя преждевременный стоп-кодон.
Однако за счет формирования в интроне устойчивой вторичной структуры транскрипты избегают NMD и при необходимости кодируют укороченные, но функционально активные белки~\cite{Mamon2013,Golubkova2012}.

Эволюционные вариации семейства \textit{Nxf} можно разделить на три основные группы:

\begin{spacing}{0.5}
\begin{itemize}
  \item \textbf{Позвоночные.} У генов \textit{Nxf1} интрон располагается между 10-м и 11-м экзонами. Вставка содержит несколько консервативных мотивов, включая фрагмент, похожий на CTE (cis-acting transport element), необходимый для экспорта частично сплайсированных РНК.
  \item \textbf{Дрозофилиды.} У представителей рода Drosophila кассетный интрон локализуется между 5-м и 6-м экзонами. В нем отсутствуют длинные гомологичные вставки, но присутствуют два тракта, обогащенные аденином (А), формирующие прочную вторичную структуру и защищающие транскрипт от деградации в ядре~\cite{Mamon2013,Roy2006}.
  \item \textbf{Нематоды.} У нематод интрон тоже располагается между 5-м и 6-м экзонами, но он гораздо короче и богат тимином (Т). Консервативные гомологи в таком интроне практически отсутствуют.
\end{itemize}
\end{spacing}

Эти различия отражают долгую эволюцию семейства \textit{Nxf}: от сравнительно простой короткой вставки у нематод до сложных CTE-подобных мотивов у позвоночных, подчеркивая ключевую роль кассетного интрона в посттранскрипционной регуляции~\cite{Mamon2013,Golubkova2012}.


\subsection{Структура и функции \textit{Nxf1} (гена и белка)}

Первоначально ген \textit{Nxf1} (известный как TAP у человека) был охарактеризован как ко-фактор белка Tip герпеса saimiri, отвечающий за экспорт несплайсированных и частично сплайсированных ретровирусных мРНК путем распознавания CTE-структуры в их последовательностях~\cite{Zolotukhin2001}.
В Drosophila аналог этого гена называется \textit{sbr}.
Белок sbr включает несколько функциональных доменов: RBD (домен связывания РНК), четыре лейцин-обогащенных повтора (LRR), NTF2-подобный домен, UBA-подобный домен и сигналы ядерной локализации (NLS).
В совокупности эти домены обеспечивают узнавание мРНК и взаимодействие с компонентами ядерного порового комплекса, делая \textit{Nxf1} основным экспортером мРНК~\cite{Herold2000,Mamon2013}.

Кассетный интрон, встроенный между сегментами RBD+LRR и NTF2L+UBA, выступает в роли ``переключателя``.
При его сохранении в пре-мРНК формируется вторичная структура, позволяющая транскрипту избегать NMD и синтезировать укороченные вариации белка, обладающие специфической активностью~\cite{Mamon2013,Herold2000}.

Кроме классической функции транспорта мРНК, у \textit{Drosophila melanogaster} sbr выполняет органо- и тканеспецифические задачи.
В сперматогенных клетках ген продуцирует укороченную форму \textit{sbr}, необходимую для нормального сперматогенеза: без нее наблюдается резкое снижение фертильности~\cite{Ginanova2016}.
В центральной нервной системе sbr участвует в формировании границ между областями мозгового вещества зрительной системы, локализуясь в специфических нейронах и глиальных клетках и регулируя их ядерно–цитоплазматические комплексы~\cite{Mamon2021}.

Аналогичные эволюционно значимые особенности кассетного интрона отмечены у летучих мышей (Chiroptera).
Сравнительный анализ показал, что во многих видах этих млекопитающих в интроне \textit{Nxf1} появлялись новые вторичные структуры, связанные с тканеспецифической регуляцией экспрессии, что свидетельствует об адаптивной роли этой вставки~\cite{Bondaruk2022}.

Таким образом, \textit{Nxf1}/\textit{sbr} представляет собой пример многофункционального белка, чья доменная организация и альтернативные формы позволяют выполнять как базовую задачу — экспорт мРНК, так и специализированные функции в разных тканях и таксонах.
