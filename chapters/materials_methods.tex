\newpage
\section{Материалы и методы}

В качестве материалов выступали нуклеотидные и белковые последовательности гена \textit{Nxf1} из открытых баз данных NCBI~\cite{ncbi_general}.
Всего было проанализировано X видов, относящихся к разнообразным филогенетическим таксонам.

Поиск гена интереса для разведывательного анализа проводился внутри веб-сервиса NCBI~\cite{ncbi_general}.

Полученные данные в большинстве случаев анализировались с помощью пакета pandas v2.2.3~\cite{pandas} для языка программирования Python v3.12.6~\cite{python_3_12}.
Большинство этапов последующего анализа реализовано в виде отдельных скриптов, разработанных в рамках данной работы, если не указано другое.
Для логического разделения на блоки был использован Jupyter Notebook v1.1.1~\cite{jupyter_notebook}.

Получение данных для определенных клад производилось с помощью пакета NCBI E-utilities из BioPython v1.85~\cite{biopython} и NCBI Datasets Command-Line Interface (CLI) v18.0.2~\cite{datasets}.
Для увеличения выборок у конкретных таксономических групп был использован PSI-BLAST~\cite{psi_blast}.
Парсинг результатов, полученных с помощью PSI-BLAST, осуществлялся пакетом BioPython~\cite{biopython} и специально разработанными скриптами.
Множественные выравнивания проводились, используя алгоритм MAFFT~\cite{mafft} в программе Unipro UGENE v52.0~\cite{ugene}.

С помощью пакета инструментов MEME Suite v5.5.8~\cite{meme} проводился поиск консервативных мотивов внутри интрона.
Найденные мотивы анализировались с помощью Tomtom~\cite{tomtom} из того же пакета на базе данных Vertebrates (In vivo and in silico).

Построение вторичных структур РНК и выделение интересующих участков цветом осуществлялось локально с помощью инструмента RNAfold v2.7.0 из пакета ViennaRNA~\cite{viennarna}.

Анализ ``силы сайтов сплайсинга`` проводился локально в программе Max\-Ent\-Scan~\cite{maxentsccan}.

Филогенетический анализ включал построение деревьев с помощью IQ-TREE v2.4.0~\cite{iqtree2} и их визуализацию, используя Figtree v1.4.4~\cite{figtree}.

Работа проводилась в виртуальном окружении Mamba v1.5.5~\cite{mamba}, использованные пакеты, их версии и примеры анализа в Jupyter Notebooks можно найти в GitHub~\cite{github_general} репозитории автора: \url{https://github.com/ArtemVaska/Diploma}.

Для написания ВКР была использована система верстки LaTeX v4.76~\cite{latex}.
Все шаги анализа проводились на базе операционной системы Linux Ubuntu 22.04~\cite{ubuntu}.
