\clearpage
\section{Аннотация}

Основной функцией белкового продукта гена \textit{Nxf1} (nuclear export factor-1) является перенос мРНК из ядра в цитоплазму. Данный ген входит в семейство \textit{Nxf} и присутствует у всех описанных Opisthokonta. В составе его нуклеотидной последовательности присутствует эволюционно-консервативный блок, состоящий из двух экзонов и интрона между ними. В процессе альтернативного сплайсинга возможно образование интрон-содержащего транскрипта с преждевременным стоп-кодоном в этом интроне. Примечательно, что такой транскрипт может избегать нонсенс-опосредованного распада, и в результате может синтезироваться укороченный белок со своими уникальными функциями.

Работа посвящена изучению структуры гена \textit{Nxf1} у представителей различных филогенетических групп животных, среди которых наибольшее внимание было уделено Actynopterygii. Главной целью является выявление закономерностей эволюции генов семейства \textit{Nxf} и подробный анализ упомянутого ранее эволюционно-консервативного блока.

Было показано, что структура консервативного блока имеет определенные характеристики, которые меняются в пределах разных таксономических ветвей. Вариативность характеристик больше у тех групп, которые дальше эволюционно друг от друга. Также было продемонстрировано, что внутри интрона из этого блока есть участки, которые способствуют формированию особой вторичной структуры интрон-содержащего транскрипта. За счет их наличия возможно сохранение такого транскрипта и последующая трансляция с синтезом укороченной формы белка.

\vspace{1em}

\textbf{Ключевые слова:} nuclear export factor (nxf), интрон-содержащие транскрипты, эволюционно-консервативные последовательности, вторичные структуры транскриптов, Actynopterygii