\clearpage
\section*{Abstract}

The main function of the protein product of the \textit{Nxf1} gene (nuclear export factor-1) is the transfer of mRNA from the nucleus to the cytoplasm. This gene is a member of the \textit{Nxf} family and is present in all described Opisthokonta. Its nucleotide sequence contains an evolutionarily conservative block consisting of two exons and an intron between them. In the process of alternative splicing, an intron-containing transcript with a premature stop codon in this intron can be formed. It is noteworthy that this transcript can avoid nonsense-mediated decay, and as a result, a shortened protein with its own unique functions can be synthesized.

The work is devoted to the study of the structure of the \textit{Nxf1} gene in representatives of various phylogenetic groups of animals, among which the greatest attention was paid to Actynopterygii. The main goal is to identify patterns of evolution of the \textit{Nxf} family genes and do a detailed analysis of the previously mentioned evolutionarily conservative block.

It was shown that the structure of the conservative block has certain characteristics that change within different taxonomic branches. The variability of characteristics is greater in those groups that are further evolutionarily from each other. It was also demonstrated that within the intron of this block there are areas that contribute to the formation of a special secondary structure of the intron-containing transcript. Due to their presence, it is possible to preserve such a transcript and subsequent translation with the synthesis of a shortened form of the protein.

\vspace{1em}

\textbf{Keywords:} nuclear export factor (nxf), intron-containing transcripts, evolutionarily conservative sequences, secondary structures of transcripts, Actynopterygii