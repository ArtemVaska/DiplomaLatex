\clearpage
\section{Введение}

Для большинства генов высших эукариот характерна мозаичная структура, в составе которой выделяют кодирующие участки — экзоны и некодирующие — интроны.
В процессе созревания транскрипта интроны, как правило, вырезаются в процессе сплайсинга, и из ядра выходит мРНК, лишенная интронных последовательностей.
Однако альтернативный сплайсинг позволяет получать несколько различных зрелых мРНК из одной пре-мРНК, что значительно расширяет протеом без увеличения числа генов.

Особый интерес представляют транскрипты, сохраняющие интрон (intron reten\-tion, IR).
Как правило, в таком интроне присутствует преждевременный стоп-кодон (pre\-mature termination codon, PTC), поэтому существует специальный механизм для проверки качества транскриптов перед выходом из ядра — нонсенс-опосредованный распад мРНК (nonsense mediated mRNA decay, NMD), который препятствует выходу таких транскриптов в цитоплазму.
Однако, несмотря на наличие специфического механизма, среди различных групп, эволюционно далеких друг от друга, описаны случаи существования транскриптов с сохраненным интроном.
Отдельно можно выделить дрозофилу и человека, для которых известно семейство генов \textit{Nxf} (nuclear export factor), в котором нас заинтересовал ген \textit{Nxf1}.
Данный ген кодирует белок, являющийся основным транспортером мРНК из ядра в цитоплазму.

В состав последовательности гена \textit{Nxf1} (nuclear export factor 1) входит так называемая ``консервативная кассета``, которая включает два коротких экзона размером 110 и 37 нуклеотидов в каноническом варианте и ``кассетный`` интрон между ними.
Названия сформулированы нашей научной группой и будут использоваться в дальнейшем повествовании.
Эта структура сохраняется также и у представителей других филогенетических групп.
Благодаря образованию специфической вторичной структуры или наличию в последовательности интрона специфических последовательностей, например конститутивного транспортного элемента (constitutive transport element, CTE), транскрипт, содержащий преждевременный стоп-кодон, избегает NMD и может кодировать укороченную форму белка.

Анализ подобных транскриптов показал, что консервативные элементы ``кассеты`` \textit{Nxf1} специфичны для разных клад организмов, а интрон-содержащие транскрипты формируют уникальные вторичные структуры, что подчеркивает эволюционную и функциональную значимость интронов.

Научная новизна работы заключается в сравнительном анализе структуры гена \textit{Nxf1} у представителей различных филогенетических групп, данных по которым ранее не было, с целью выявления закономерностей эволюции нуклеотидной последовательности гена \textit{Nxf1} и его белковых продуктов.

В бакалаврской работе было показано, что ``консервативная кассета`` сохраняет свойства внутри артропод, особенно внутри семейства Drosophilidae, однако вопрос о степени консервативности и специфике структурных элементов у более широкого круга организмов остается открытым.
Помимо сравнительного анализа последовательностей, важной частью исследования является построение вторичных структур интрон-содержащих транскриптов и выявление консервативных мотивов внутри интрона, способствующих его сохранению и избеганию нонсенс-опосредованного распада.


\subsection{Цель работы}

Изучить структуру гена \textit{Nxf1} у представителей разных филогенетических групп животных для выявления эволюционных закономерностей и особенностей ``кассетной`` структуры, а также проанализировать вторичные структуры интрон-содержа\-щих транскриптов.


\subsection{Задачи}

\begin{enumerate}[left=\parindent]
  \item Найти нуклеотидные и аминокислотные последовательности гена \textit{Nxf1} у различных групп животных.
  \item Произвести поиск ``консервативной кассеты`` в нуклеотидной последовательности гена у найденных организмов.
  \item Выполнить анализ структуры ``консервативной кассеты``, сравнить полученные последовательности между собой.
  \item Выявить и охарактеризовать консервативные участки ``кассетного`` интрона и прилегающих экзонов у видов из исследуемых таксонов.
  \item Провести анализ вторичной структуры интрон-содержащих транскриптов и оценить консервативные мотивы внутри интрона, потенциально способствующие его сохранению.
\end{enumerate}
