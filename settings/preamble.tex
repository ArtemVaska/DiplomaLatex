% Макет страницы
\usepackage[a4paper, left=3.5cm, right=1.5cm, top=2cm, bottom=2cm]{geometry} % подключает пакет geometry для управления размерами страницы
\usepackage[onehalfspacing]{setspace} % 1.5 интервал
\usepackage{indentfirst} % добавляет отступ для абзаца после секции

% Настрока заголовков и подзаголовков для оглавления
\usepackage{titlesec}

% Настройка \section: шрифт 14pt, выравнивание по центру
\titleformat{\section}
  {\normalfont\fontsize{14}{16}\selectfont\bfseries\centering} % стиль шрифта заголовка
  {} % без номера в тексте, но номер будет в ToC
  {0pt} % расстояние между номером и заголовком
  {} % код перед названием секции
% Отступ слева = абзацный отступ, сверху = 2 строки, снизу = 1 строка
\titlespacing*{\section}
  {0pt} % левый отступ не нужен
  {2ex} % отступ сверху
  {1ex} % отступ снизу

% Настройка \subsection: шрифт 12pt, выравнивание по абзацу
\titleformat{\subsection}
  {\normalfont\fontsize{12}{14}\selectfont\bfseries} % стиль шрифта подзаголовка
  {} % без номера в тексте, но будет номер в ToC
  {0pt} % расстояние между номером и заголовком
  {} % код перед названием

\titlespacing*{\subsection}
  {\parindent} % левый отступ как у абзаца
  {3ex} % отступ сверху
  {1ex} % отступ снизу

% Добавляем точки в оглавление
\usepackage{tocloft}
\renewcommand{\cftsecleader}{\cftdotfill{\cftdotsep}} % добавляем точки
\renewcommand{\cftdotsep}{1} % плотность точек

% Кодировка и поддержка кириллицы
\usepackage[utf8]{inputenc} % позволяет использовать текст в кодировке UTF-8
\usepackage[T2A]{fontenc} % поддержка русских шрифтов (кириллица)
\usepackage[english, russian]{babel} % включение правил типографики для английского языка (переносы, заголовки и т.д.)

% Кавычки и цитаты
\usepackage{csquotes} % поддержка корректного форматирования кавычек и цитат (особенно важно при использовании biblatex)

% Вставка изображений
\usepackage{graphicx} % позволяет вставлять и масштабировать изображения (\includegraphics)

% Указание папки для поиска изображений
\graphicspath{{images/}}

% Таблицы
\usepackage{array} % расширенные возможности для работы с таблицами
\usepackage{tabularx} % автоматический подбор ширины столбцов
\usepackage{dcolumn} % выравнивание чисел по разделителю

% Математические выражения
\usepackage{amsmath, amssymb} % расширенные математические окружения; дополнительные математические символы

% Ссылки
\usepackage[
    colorlinks=true, % включение цвета для ссылок
    linkcolor=black, % цвет оглавления
    urlcolor=blue, % цвет url-ссылок в тексте и библиографии
    citecolor=black % цвет для текста в цитатах
]{hyperref} % добавляет активные гиперссылки на разделы, рисунки, библиографию и внешние URL

% Подписи
\usepackage{caption} % более гибкое управление подписями к рисункам и таблицам
\usepackage{subcaption} % позволяет делать несколько подрисунков с подписями (a), (b), ...

% Библиография
\usepackage[backend=biber, style=gost-numeric, autolang=other]{biblatex}
% использование biber для обработки ссылок
% ГОСТ-стиль нумерованной библиографии
% язык библиографии подбирается автоматически, нужно поле langid = {russian} или langid = {english} в .bib файле
\addbibresource{bib/bibliography.bib}